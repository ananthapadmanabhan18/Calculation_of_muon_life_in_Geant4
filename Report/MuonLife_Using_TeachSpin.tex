\documentclass{article}

\usepackage[utf8]{inputenc}
% \usepackage[utf8]{inputenc}
\usepackage{multicol}
\usepackage{dcolumn}
\usepackage[a4paper,top=3cm,bottom=3cm,left=1.5cm,right=1.5cm,marginparwidth=1.75cm]{geometry}
\usepackage{multicol}
\setlength{\columnsep}{0.5cm}
\usepackage{multirow}
\usepackage{amsmath}
\usepackage{graphicx}
\usepackage{hyperref}
\hypersetup{colorlinks=true,linkcolor=blue,filecolor=magenta,urlcolor=cyan,}
\usepackage{amsfonts}
\usepackage{mathtools}
\usepackage{lipsum}
\usepackage{float}
\usepackage{layout}
\usepackage{bm}

\usepackage{listings}
\usepackage{xcolor}
\definecolor{codegreen}{rgb}{0,0.6,0}
\definecolor{codegray}{rgb}{0.5,0.5,0.5}
\definecolor{codepurple}{rgb}{0.58,0,0.82}
\definecolor{backcolour}{rgb}{0.95,0.95,0.92}
\lstdefinestyle{mystyle}{
    backgroundcolor=\color{backcolour},
    commentstyle=\color{codegreen},
    keywordstyle=\color{blue},
    numberstyle=\tiny\color{codegray},
    stringstyle=\color{codepurple},
    basicstyle=\ttfamily\footnotesize,
    breakatwhitespace=false,
    breaklines=true,
    captionpos=b,
    keepspaces=true,
    numbers=left,
    numbersep=5pt,
    showspaces=false,
    showstringspaces=false,
    showtabs=false,
    tabsize=2
}
\lstset{style=mystyle}





\begin{document}










%TitlePage%TitlePage%TitlePage%TitlePage%TitlePage%TitlePage%TitlePage%TitlePage%TitlePage%TitlePage%TitlePage%TitlePage%TitlePage
\thispagestyle{empty}
\baselineskip25pt
\begin{center}
{\Large {\textbf{CALCULATION OF MUON LIFE TIME USING SCINTILLATION \\AND CHERENKOV RADIATION WITH TEACHSPIN SETUP}}}\\
\end{center}

\vfill
\baselineskip15pt
\begin{center}
{\em Open Lab Report Submitted} \\
in Partial Fulfilment of the Requirements \\
for the course \
\vskip .30\baselineskip
{\large{\bf P442-Integrated Laboratory-II}}
\end{center}
\baselineskip25pt

\vfill
\begin{center} {\bf {\em by}} \\
{\large{\bf ANANTHA PADMANABHAN M NAIR}} \\
Course Instructors:\\
\textbf{Dr. Ashok Mohapatra\\
Dr. Gunda Santosh Babu}
\end{center}

\vfill
\begin{center}
\begin{figure}[h!]
\centering
\includegraphics[scale=0.2]{Images/logo1.jpg}
\end{figure}
 {\bf {\em to the }} \\
{\bf {\large School of Physical Sciences}} \\
{\bf {\large National Institute of Science Education and Research}} \\
{\bf Bhubaneswar} \\
{\bf \today} 
\end{center}
%TitlePage%TitlePage%TitlePage%TitlePage%TitlePage%TitlePage%TitlePage%TitlePage%TitlePage%TitlePage%TitlePage%TitlePage%TitlePage%










%Content Table Page%Content Table Page%Content Table Page%Content Table Page%Content Table Page%Content Table Page
\pagenumbering{roman}
\newpage
\newgeometry{top=2.5cm,bottom=2.5cm,left=3.5cm,right=3.5cm}
\begin{center}
 \tableofcontents  
 \newpage
 \listoffigures
 \listoftables 
\end{center}
\restoregeometry
%Content Table Page%Content Table Page%Content Table Page%Content Table Page%Content Table Page%Content Table Page



\newpage
\begin{center}
    \large{\textbf{CALCULATION OF MUON LIFE TIME USING SCINTILLATOR DETECTOR \\ AND CHERENKOV RADIATION WITH TEACHSPIN SETUP}}
\end{center}
\begin{abstract}
    This experiment aimed to determine the mean life of muons using the teachspin setup. Initially, data was collected with a scintillation detector connected to a PMT using teachspin software. Later, the setup was modified, replacing the scintillation detector with a Cherenkov detector connected to the PMT. Data was collected using the same software, and the mean muon life was calculated from this data. Subsequently, both sets of data were analyzed using Python and ROOT libraries. Additionally, a Geant4 simulation was conducted for the modified setup, and its data was also analyzed using ROOT libraries. A comparison was made between the results obtained from the Geant4 simulation and the teachspin setup, followed by a discussion of the findings.
\end{abstract}
\begin{multicols}{2}
\pagenumbering{arabic}






\section{\label{intro}Introduction}
Muon decay experiments have a rich history dating back to the mid-20th century when pioneering physicists first began studying these elusive subatomic particles. Muons, which are heavier cousins of electrons, are known for their fleeting existence, making them fascinating subjects of study in particle physics. Their relatively short lifespan, coupled with their ability to penetrate matter, has led to a myriad of experimental investigations aimed at understanding fundamental aspects of particle physics.

Over the years, muon decay experiments have played a crucial role in advancing our understanding of the Standard Model of particle physics, providing valuable insights into the nature of fundamental particles and their interactions. These experiments have not only contributed to our knowledge of particle physics but have also found practical applications in various fields, including astrophysics, cosmology, and even archaeology.
%%%%%%%%%%%%%%%%%%%%%%%%%%%%%%%%%%%%%%%%%%%%%%%%%%%%%%%%%%%%%%%%%%%%%%%%%%%%%%%%%%%%%%%%%%%%%%%%%%%%%%%%%%%%%%%%%%%%%%%%%%%%%%%%%%%%%%%%%%%%%%%%%%%%%%%%%%
\section{\label{theory}Theory}
Muons are elementary particles similar to electrons but with a much greater mass, making them key players in the study of particle physics. They are classified as leptons, belonging to the same family as electrons and neutrinos. Muons are unstable, with a mean lifetime of around 2.2 microseconds, and they decay into lighter particles, predominantly electrons and neutrinos, through weak interactions. Despite their short lifespan, muons are abundant in cosmic rays and are also produced in particle accelerators, providing ample opportunities for experimental study. 

\subsection{Cosmic Muons}
The muons are primarily generated by the interactions of cosmic rays with the atmospheric particles. The cosmic rays are high-energy particles originating from outer space, which interact with the Earth's atmosphere, producing a cascade of secondary particles, including muons. 

\subsubsection{Cosmic Rays}
Cosmic rays are high-energy particles that originate from various sources beyond our solar system, such as distant stars, supernovae, and active galactic nuclei. They consist of  protons(87\%),Alpha Particles(12\%), and atomic Heavy Nuclei (1\%), some of which can have energies millions or even billions of times greater than those produced in the most powerful particle accelerators on Earth. When  cosmic rays enter the Earth's atmosphere, they interact with air molecules, producing a cascade  of secondary particles, including muons, neutrinos, and gamma rays.

\subsection{Muon Production}

When there Cosmic Rays Reach the Earths atmosphere, it collides with the air molecules and produces 
many particles, mostly Pions and Kaons. These are the primary Particles. These particles then decay to produce a wide variety
of particles most of which are muons. Also, more than 90\% of the cosmic muons are produced from Pions.

From the Kaons, We can see that the muons are produced by the weak interactions and it also produces neutrinos given by:
\begin{equation}
    K^{+} \longrightarrow \mu^{+} + \nu_{\mu}
\end{equation}

From the decay of Pions, 64\% of the time, disintegrates directly into $\mu^+$; 21\% of the time into $\pi^0$and$\pi^+$ and Only 6\% of the disintegrations produce three particles-$\mu^+$,$\mu^+$,$\mu-$. These pions then decay to produce muons according to:
\begin{equation}
    \pi^{+} \longrightarrow \mu^{+} + \nu_{\mu}
\end{equation}
\begin{equation}
    \pi^{-} \longrightarrow \mu^{-} + \Bar{\nu_{\mu}}
\end{equation}



While the decay of these primary particles produces mostly muons, Many other elementary particles are also produced
within this process. During the process of generation of primary particles that is the Kaons and pions, $K^0$ and $\pi^0$ are also produced
which on decay produces $\gamma$-particles. These Muons and other particles further decay to produce Electrons and Positrons.





\subsection{Muon Properties}

Muons are unstable particles having intermediate mass between that of an 
electron and a proton, just like charged pions. Compared to pions, they are 
a little lighter.The muons carry one unit of electrical charge, either positive 
or negative, and are electrically charged. The life time of the muon is $2.2\mu s$.
The properties of the muons are tabulated below in Table-\ref{muonproperty}

\begin{table}[H]
    \centering
    \resizebox{0.85\columnwidth}{!}{%
    \begin{tabular}{|cc|c|}
        \hline
        \multicolumn{2}{|c|}{\textbf{Properties}}                           & \textbf{Values}              \\ \hline
        \multicolumn{1}{|c|}{\multirow{2}{*}{\textbf{Mass}}} & $m_{\mu}$    & $206.7686m_e$                \\ \cline{2-3} 
        \multicolumn{1}{|c|}{}                               & $m_{\mu}c^2$ & 105.659MeV                   \\ \hline
        \multicolumn{1}{|c|}{\textbf{Mean Life}}             & $\tau_{\mu}$ & $2.197 \mu s$                \\ \hline
        \multicolumn{1}{|c|}{\textbf{Spin}}                  & $s_{\mu}$    & 1/2                          \\ \hline
        \multicolumn{1}{|c|}{\textbf{Magnetic   Moment}}     & $\mu_{\mu}$  & $\frac{eh}{4\pi   m_{\mu} }$ \\ \hline
    \end{tabular}%
    }
    \caption{Properties of Muons}
    \label{muonproperty}
\end{table}



\subsection{Life time of Muons at Relativistic Speeds}

We know that the average energy of the cosmic muons are $4GeV$, which, for a particle of mass $m_{\mu}$, is a relativistic speed. The life time of the muon at rest is $2.2\mu s$. But when the muon is moving at relativistic speeds (with respect to the observer on earth), the life time of the muon is given by the formula:
\begin{equation}
    \tau = \frac{\tau_{0}}{\sqrt{1-\frac{v^2}{c^2}}}
\end{equation}

Where $\tau_{0}$ is the life time of the muon at rest, $v$ is the velocity of the muon and $c$ is the speed of light. In order to calculate the $\gamma$ factor, we can use the formula:
\begin{equation}
    E_{total} = 4GeV = \gamma m_{\mu}c^2
\end{equation}

From this, we can calculate the $\gamma$ factor and then calculate the life time of the muon at relativistic speeds. The gamma came out to be 38.1 and the life time of the muon at relativistic speeds is $ 84 \mu s$ which is more than eneoygh time for the high energy muons to reach the earths surface.


\subsection{Decay of Muons}

These Muons decay into electrons and neutrinos through the weak interactions. The decay of the muon is given by:
\begin{equation}
    \mu^{-} \longrightarrow e^{-} + \Bar{\nu_{e}} + \nu_{\mu}
\end{equation}
and for the muon decay into positron, the decay is given by:
\begin{equation}
    \mu^{+} \longrightarrow e^{+} + \nu_{e} + \Bar{\nu_{\mu}}
\end{equation}


The decay of the muon in the scintillation detector occurs when the muon comes to a complete stop inside the detector. This muon spends some time in the order of micro seconds inside the detector before decaying into electrons and neutrinos. After coming to rest, the muons only energy is its mass energy $m_\mu c^2$ whih gets converted to the kinetic energy of the electrons and neutrinos. The electrons and neutrinos are produced with a wide range of energies, but the average energy of the electrons is $35 MeV$(The two neutrinos, $\nu$, carry away the rest of the energy undetectably.). This electron has eneough energy to excite the scintillator material and produce light. So, we will get 2 pulses from the scintillator, one due to the muon stopping and the other due to the decay of the muon.


Indeed, the distribution decaying over time should follow an exponential pattern, representing the 'survival curve' of muons. This curve won't depict the muons' survival time since their inception in the upper atmosphere, but rather their survival time since reaching the scintillator.

So, we can get the mean life by fitting the distribution of the decay times of the muons in the scintillator detector. The fitting function is given by:
\begin{equation}
    N(t) = N_0 e^{-t/\tau} + B
\end{equation}
Where $N(t)$ is the number of muons decaying at time $t$, $N_0$ is the number of muons decaying at time $t=0$, $\tau$ is the mean life of the muon and $B$ is the background noise. The mean life of the muon can be calculated by fitting the distribution of the decay times of the muons in the scintillator detector.


Now, from the calculations from previous subsection, the time required for the muon to reach the scintillator detector is approximately $200 \mu s$. 




The fraction of the muons that survive to get to the scintillator detector is given by:
\begin{equation}
    f = e^{-(t=200 \mu s /(\gamma = 40) )/(\tau = 2 \mu s)} = e^{-5/2}
\end{equation}
which is approximately 10\% od the total muons.
So we will be able to detect approximately 10\% of all the muons that enter the earth with a flux of $1$ per $cm^2$ per second.




\subsection{Geant4 Simulations}

Geant4 is a powerful toolkit developed by CERN for simulating the passage of particles through matter. Widely used in various fields including high-energy physics, medical physics, and space science, Geant4 offers a comprehensive set of functionalities for modeling complex geometries, particle interactions, and detector responses. Its versatility allows researchers to simulate a wide range of experimental setups accurately, aiding in the design, optimization, and analysis of experiments. With its open-source nature and continuous development, Geant4 remains at the forefront of particle transport simulations, facilitating groundbreaking discoveries and advancements in numerous scientific disciplines.

The Simulations with an almost same geometry as that is present in the lab has been done in the Geant4 and compared with the results obtained from the teachspin setup. The Geant4 simulations were done with the same geometry as that of the teachspin setup and the data was analyzed using the ROOT libraries.This is done for both the setup with the old scintillator detector and the new Cherenkov detector.


%%%%%%%%%%%%%%%%%%%%%%%%%%%%%%%%%%%%%%%%%%%%%%%%%%%%%%%%%%%%%%%%%%%%%%%%%%%%%
%%%%%%%%%%%%%%%%%%%%%%%%%%%%%%%%%%%%%%%%%%%%%%%%%%%%%%%%%%%%%%%%%%%%%%%%%%%%%
\section{\label{expsetup}Experimental Setup and Procedure }

%%%%%%%%%%%%%%%%%%%%%%%%%%%%%%%%%%%%%%%%%%%%%%%%%%%%%%%%%%%%%%%%%%%%%%%%%%%%%
\section{\label{observations}Observations and Data}

%%%%%%%%%%%%%%%%%%%%%%%%%%%%%%%%%%%%%%%%%%%%%%%%%%%%%%%%%%%%%%%%%%%%%%%%%%%%%
%%%%%%%%%%%%%%%%%%%%%%%%%%%%%%%%%%%%%%%%%%%%%%%%%%%%%%%%%%%%%%%%%%%%%%%%%%%%%
\section{\label{dataanalysis}Data analysis}
%%%%%%%%%%%%%%%%%%%%%%%%%%%%%%%%%%%%%%%%%%%%%%%%%%%%%%%%%%%%%%%%%%%%%%%%%%%%%
%%%%%%%%%%%%%%%%%%%%%%%%%%%%%%%%%%%%%%%%%%%%%%%%%%%%%%%%%%%%%%%%%%%%%%%%%%%%%
\section{\label{error}Error Analysis}

%%%%%%%%%%%%%%%%%%%%%%%%%%%%%%%%%%%%%%%%%%%%%%%%%%%%%%%%%%%%%%%%%%%%%%%%%%%%%
%%%%%%%%%%%%%%%%%%%%%%%%%%%%%%%%%%%%%%%%%%%%%%%%%%%%%%%%%%%%%%%%%%%%%%%%%%%%%
\section{\label{results}Results}

%%%%%%%%%%%%%%%%%%%%%%%%%%%%%%%%%%%%%%%%%%%%%%%%%%%%%%%%%%%%%%%%%%%%%%%%%%%%%
%%%%%%%%%%%%%%%%%%%%%%%%%%%%%%%%%%%%%%%%%%%%%%%%%%%%%%%%%%%%%%%%%%%%%%%%%%%%%
\section{\label{Conclusion}Conclusion and Discussions}

hihi-\cite{ROOT}
%%%%%%%%%%%%%%%%%%%%%%%%%%%%%%%%%%%%%%%%%%%%%%%%%%%%%%%%%%%%%%%%%%%%%%%%%%%%%
%%%%%%%%%%%%%%%%%%%%%%%%%%%%%%%%%%%%%%%%%%%%%%%%%%%%%%%%%%%%%%%%%%%%%%%%%%%%%

\end{multicols}
\bibliographystyle{plain}
\bibliography{bib.bib}


\end{document}